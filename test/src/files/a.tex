\documentclass[10pt]{article}
\usepackage[margin=1.25in]{geometry}
\usepackage{amsmath, amssymb}
\usepackage{graphicx}
\usepackage{fancyhdr}
\usepackage{tikz, pgfplots}
\usetikzlibrary{positioning}
\usepackage{float}


\pagestyle{fancy}
\fancyhf{}
\rhead{David Lu}
\lhead{A6}
\setlength{\jot}{6pt}

\begin{document}
\begin{enumerate}
    \item 
    \begin{enumerate}
        \item The base case is $P(2)$. They start here because $1$ is neither composite nor prime, and if $P(1)$ was considered in the induction, then the proof wouldn't work as you can have an infinite different number of powers of one to create different factorization of the same number.
        \item
        When they define $m = (k+1)/p_1$, they use the fact that prime numbers are greater than one because otherwise, $m$ would be greater than $k+1$.
        \item
        When they use the inductive hypothesis to say $m$ can be written as the product of primes, $m$ never equals to $k+1-1$, so regular induction will not work. Therefore, they must use strong induction.
        \item
        Since $p_2$ is an integer, by definition, $p_1 \mid q_1q_2$. Then by Euclid's Lemma $p_1 \mid q_1 \vee p_1 \mid q_2$. Next, assume $p_1 \mid q_1$ and $p_1 \nmid q_2$. If $p_1 \nmid q_1$, then rearrange $q_1, q_2$ so that it does. Since $q_1$ is prime, its only factors are $1$ or $q_1$, but since $p_1$ is a prime but $1$ is not a prime, $p_1$ must equal $q_1$. Next, dividing the equation $p_1p_2 = q_1q_2$ by either $p_1$ or $p_2$ yields $p_2 = q_2. \square$
    \end{enumerate}
    \item
    Let $a,b$ be arbitrary integers. Assume that $\gcd{(a + 3b, 5ab)} = 1$. 
    Well, by BL, $\exists x,y \in \mathbb{Z}$ such that $(a+3b)x + (5ab)y = 1$. Next, rearranging: 
    \begin{align*}
        (a + 3b)x + (5ab)y &= 1\\
        ax + 3bx + 5aby &= 1\\
        (x+5by)a + (3x)b &=1
    \end{align*}
    Since $x+5by$ and $3x$ are both integers divisible by 1, by GCDCT, $\gcd{(a,b)} = 1$. $\square$
    
    \item
    Let $p$ be an arbitrary prime number. Let $s,t$ be arbitrary natural numbers such that $s,t < p$. Well, since $s,t \neq p$, and $p$ has no factors other than $p$ or $1$, by the CCT, $\gcd{(p,s)} = 1 = \gcd{(p,t)}$. Then, by BL, $\exists x_1,y_1,x_2,y_2 \in \mathbb{Z}$ such that $px_1 + sy_1 = 1$ and $px_2 + ty_2 = 1$. Multiplying these two equations together: 
    \begin{align*}
        (px_1 + sy_1)(px_2 + ty_2) &= (1)(1) \\
        p^2x_1x_2 + px_1ty_2 + sy_1px_2+sty_1y_2 &= 1\\
        p(px_1x_2 + x_1ty_2 + sy_1x_2) + st(y_1y_2) &= 1 \\
    \end{align*}
    Since, $(px_1x_2 + x_1ty_2 + sy_1x_2)$ and $(y_1y_2)$ are both integers divisible by 1, by the GCDCT, $\gcd{(p, st)} = 1$.  $\square$

    \item
    Note that $42^{42} = (2\cdot 3\cdot 7)^{42} = 2^{42}\cdot 3^{42} \cdot 5^0 \cdot 7^{42}$. Also, note that $8!^8 = (2^7 \cdot 3^2 \cdot 5 \cdot 7)^8 = 2^{56} \cdot 3^{16} \cdot 5^8 \cdot 7^8$. Then by the GCD PF, $\gcd{(42^{42}, 8!)} = 2^{42} \cdot 3^{16} \cdot 5^0 \cdot 7^8 = 2^{42} \cdot 3^{16} \cdot 7^8$
    \item
    By the UFT, 
    $$a=p_1^{\alpha_1}\cdots p_r^{\alpha_r},\qquad b=p_1^{\beta_1}\cdots p_r^{\beta_r}$$
    Where $p_i$ are unique primes and $\alpha_i, \beta_i >=0$. Allow the exponents to equal $0$ if $p_i$ occurs in one prime but not the other. We also know that $5 \nmid a$, which means we can re-write $a$ and $b$ as 
    $$a = p_1^{\alpha_1}\cdots p_r^{\alpha_r} \cdot 5^0, \qquad b=p_1^{\beta_1}\cdots p_r^{\beta_r} \cdot 5^{\beta_k}$$
    Next, by the GCDPF,
    $\gcd{(a,b)} = p_1^{\gamma_1}\cdots p_r^{\gamma_r} \cdot 5^0$, where $\gamma_r$ = $\min{(\alpha_r, \beta_r)}$. We also note that that by the Euclidean algorithm $\gcd{(a, a+5b)} = \gcd{(a, a+5b-a)} = \gcd{(a, 5b)}$. We can now write $5b$ as
        $$\qquad 5b=p_1^{\beta_1}\cdots p_r^{\beta_r} \cdot 5^{\beta_k+1}$$
    Then, again by GCDPF, 
    
    \begin{align*}
        gcd{(a, 5b)} &= p_1^{\min{(\alpha_1, \beta_1)}} \cdots p_r^{\min{(\alpha_r, \beta_r)}} \cdot 5^{\min{(0, \beta_k + 1)}} \\
                    &=  p_1^{\gamma_1} \cdots p_r^{\gamma_r} \cdot 5^0\\
                    &= \gcd{(a, b)} \qquad \square
    \end{align*} 
    \item
    \begin{enumerate}
\item
    \underline{First, prove there's at least one solution:}\\
    By DA, $n = pq + r$ for some integers $q, r$ where $0 \leq r < p$. Then we must find a $k \in S$ such that $p \mid pq + r + k$. \\ \\
    \underline{Case 1: $r = 0$}\\
    If $r = 0$, then consider $k = 0$. Then $n + k = pq + 0 + 0= pq$. Since $q \in \mathbb{Z}$, then by definition $p \mid pq$ which implies $p \mid n + k$.\\
    \underline{Case 2: $r > 0$}\\
    If $r > 0$, then consider $k = p - r$. Then, $n + k = pq + r + p - r = pq + p = p(q+1)$. Since $q+1 \in \mathbb{Z}$, by definition, $p \mid p(q+1)$ which implies $p \mid n + k$.\\
    \\
    \underline{Prove that $\forall k_1, k_2 \in \mathbb{Z}$, if $p \mid n + k_1$ and $p \mid n + k_2$, then $k_1 = k_2$:}\\
    For contradiction, assume $k_1 > k_2$. Since $p \mid n + k_1$ and $p \mid n + k_2$, by DIC, $p \mid (n+k_1) - (n+k_2) = k_1 - k_2$. This implies $\exists a \in \mathbb{Z}$ such that $pa = k_1 - k_2$. 
    \begin{align}
        & pa = k_1 - k_2 \\
        p &\leq k_1 - k_2\\
        p+k_2 &\leq k_1
    \end{align}
    But the condition that $0 \leq k_1, k_2 < p$ implies $p + k_2 > k_1$ which is a contradiction. Therefore the assumption that $k_1 > k_2$ is false which means $k_1 \leq k_2$. 
    \\ \\
    The process for showing $k_1 \geq k_2$ is similar, and therefore omitted. 
    \\ \\
    If $k_1 \leq k_2$ and $k_1 \geq k_2$ are both true, then $k_1 = k_2$. Therefore, $k$ is unique. $\square$
    \item
    From part a), there exists only one integer $k$ in the closed interval $[0, p-1]$ such that $p \mid n + k$. We also note that the $gcd{(ap, p)} = p$ for all integers $a$ and primes $p$. We also note that for all integers $x$, where $x \neq 0$ and $p \nmid x$, $\gcd{(x, p)} = 1$. These are easily proven using UFT and GCD PF, but Latex is too hard and it's 2am. Then,
    \begin{align*}
        &\prod_{i=0}^{p-1}\gcd{(n+i, p)}\\
        =&\gcd{(n+0, p)} \cdot \gcd{(n+1, p)} \cdots \gcd(n+k, p) \cdots \gcd{(n+p-1, p)}\\
        =&1 \cdot 1  \cdots p \cdots 1\\
        =& p \text{, as desired}
    \end{align*}
    \end{enumerate}
    
    
    



\end{enumerate}
\end{document}